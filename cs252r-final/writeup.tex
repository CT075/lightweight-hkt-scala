\documentclass[acmsmall,screen]{acmart}
\setcopyright{none}
\copyrightyear{2022}
\acmYear{2022}
\acmDOI{}

\citestyle{acmauthoryear}

\usepackage{listings}
\usepackage{fancyhdr}
\usepackage{mathpartir}
\usepackage{bussproofs}
\usepackage[page]{appendix}

\definecolor{dkgreen}{rgb}{0,0.6,0}
\definecolor{gray}{rgb}{0.5,0.5,0.5}
\definecolor{mauve}{rgb}{0.58,0,0.82}

\lstdefinestyle{ocaml}{
  language=[Objective]caml,
  aboveskip=3mm,
  belowskip=3mm,
  showstringspaces=false,
  columns=flexible,
  basicstyle={\small\ttfamily},
  numbers=none,
  numberstyle=\tiny\color{gray},
  keywordstyle=\color{blue},
  commentstyle=\color{dkgreen},
  stringstyle=\color{mauve},
  frame=single,
  breaklines=true,
  breakatwhitespace=true,
  tabsize=2,
  literate={'"'}{\textquotesingle "\textquotesingle}3,
  escapeinside={`}{`}
}

\lstdefinestyle{scala}{
  language=scala,
  aboveskip=3mm,
  belowskip=3mm,
  showstringspaces=false,
  columns=flexible,
  basicstyle={\small\ttfamily},
  numbers=none,
  numberstyle=\tiny\color{gray},
  keywordstyle=\color{blue},
  commentstyle=\color{dkgreen},
  stringstyle=\color{mauve},
  frame=single,
  breaklines=true,
  breakatwhitespace=true,
  tabsize=2,
  escapeinside={`}{`}
}

\newcommand{\DOTw}{\text{DOT}$^\omega$}

\begin{document}

\title{Lightweight Higher-kinded Dependent Object Types}

\author{Cameron Wong}
\email{cwong@g.harvard.edu}
\affiliation{%
  \institution{Harvard University}
  \country{USA}
}

\renewcommand{\shortauthors}{Cameron Wong}

\settopmatter{printacmref=false}
\settopmatter{printfolios=true}
\renewcommand\footnotetextcopyrightpermission[1]{}
\pagestyle{fancy}
\fancyfoot{}
\fancyfoot[R]{CS252R (Fall 2022)}
\fancypagestyle{firstfancy}{
  \fancyhead{}
  \fancyhead[R]{CS252R (Fall 2022)}
  \fancyfoot{}
}
\makeatletter
\let\@authorsaddresses\@empty
\makeatother

\begin{abstract}
  Scala, like many other functional programming languages, has first-class
  support for higher-kinded type abstraction. However, its core calculus
  Dependent Object Types (DOT) does not, complicating the design and
  implementation of the feature in modern Scala compilers.

  We demonstrate a novel encoding of higher-kinded types into Scala using only
  features known to be expressible in DOT via defunctionalization.
\end{abstract}

\keywords{
  type constructor polymorphism,
  higher-kinded types,
  higher-order genericity,
  Scala}

\maketitle
\thispagestyle{firstfancy}

\section{Introduction}

Higher-kinded types enable a greater level of abstraction by allowing code to
be generic over type \emph{constructors} rather than fully-realized types. For
example, a programmer may design a data structure that abstracts over a
container type to enable variants of the same structure to share the same
definition.

Like many other functional languages, Scala supports higher-kinded types.
However, the DOT calculus (DOT) \cite{Amin_Grutter_Odersky_Rompf_Stucki_2016}
underlying the Scala 3 Dotty compiler does not. This means that there is no
formal model of how higher-kinded types interact with Scala's other features in
full generality, making it difficult to reason about the correctness of their
implementation in Dotty, which uses an extension to DOT that has not been
proven sound \cite{OderskyDotty}.

What is so difficult about extending DOT's soundness proof to include
higher-kinded generics? While a full exploration of DOT and its metatheory is
beyond the scope of this paper, one reason is that DOT does not actually have
primitive generics at all. Instead, a parameterized type like \texttt{List[T]}
is represented with type members:
\begin{lstlisting}[style=scala]
  class List:
    type T
    // further definitions, methods, etc.
\end{lstlisting}

with specific instantiations of the parameter \texttt{T} being provided via
specialization:
\begin{lstlisting}[style=scala]
  class ListInt extends List:
    type T = Int
    // etc.
\end{lstlisting}

In this way, abstract type members can be used as a form of first-order
polymorphism, where expressions and types can refer to type variables of
kind~$*$, the kind of regular types.

Our contribution is to demonstrate an idiomatic encoding of higher-kinded
polymorphism using only such first-order features, in the hopes that this may
direct further efforts to bridge the gap between Scala and DOT.

\section{Higher-kinded types}

We motivate higher-kinded types with two examples.

\subsubsection*{Higher-kinded Data}

This example was first presented by \cite{macguire_hkd}. Consider a datatype of
\texttt{Person}, containing a name and an age. Objects in our program may be
populated by a source, such as a webform, in which individual fields may be
malformed without invalidating the data structure as a whole. However, within
the program, we would like to ensure that all \texttt{Person}s have all valid
fields. A naive modeling might use two separate classes (figure
\ref{fig::person-bad}).

\begin{figure}[ht]
  \begin{lstlisting}[style=scala]
  class Person:
    var name: String
    var age: Int

  class UnvalidatedPerson:
    var name: Option[String]
    var age: Option[Int]
  \end{lstlisting}
  \caption{Two datatypes, related only in name}\label{fig::person-bad}
\end{figure}

This is quite an unsatisfying pair of definitions. The types \texttt{Person}
and \texttt{UnvalidatedPerson} are only nominally related, meaning that if a
new field is added, \emph{both} types need to be updated, creating a
maintenance burden. Furthermore, although the relationship is obvious, any
updates will need to be done manually, without compiler assistance, introducing
the potential for human error.

Both \texttt{Person} and \texttt{UnvalidatedPerson} have the same field
names and "moral" types, differing only in the "shape" of said field type
(\texttt{T} versus \texttt{Option[T]}). This suggests that, if we could
abstract over that "shape", we could allow \texttt{Person} and
\texttt{UnvalidatedPerson} to share a skeleton \texttt{PersonImpl},
allowing the compiler to enforce the relationship between them.

\begin{figure}[ht]
  \begin{lstlisting}[style=scala]
  class PersonImpl[F[_]]:
    var name: F[String]
    var age: F[Int]

  UnvalidatedPerson = Person[Option]
  Person = Person[Id]
  \end{lstlisting}
  \caption{Unifying two nominally-related datatypes via higher-kinded
  types}
\end{figure}

Higher-kinded types provide exactly this mechanism, in which
\texttt{PersonImpl} takes a type-level function \texttt{F} and applies it to
its two fields to produce the final data structure.

\subsubsection*{Higher-ranked typeclasses}

Our second example is the venerable monad interface. A type constructor
\texttt{M} is a monad if it admits the following functions:
\begin{lstlisting}[style=scala]
  def pure[A](x: A): M[A]
  def bind[A, B](x: M[A], f: A => M[B]): M[B]
\end{lstlisting}
following the monad laws.

How can we \emph{abstract} over all possible monads? A naive attempt at
encoding the above functions as a trait might look as follows:
\begin{lstlisting}[style=scala]
  trait Monad[A]:
    def bind[B](f: A => /* ??? */): /* ??? */
\end{lstlisting}

However, an issue arises when attempting to represent the type \texttt{M[B]},
as we have no way to refer to the underlying type constructor \texttt{M}, only
the fully-applied \texttt{M[A]} implicitly as \texttt{self}.

With higher-kinded type members, however, we can solve this directly:
\begin{lstlisting}[style=scala]
  trait MonadOps:
    type M[_]
    def pure[A](x: A): M[A]
    def bind[A, B](x: M[A], f: A => M[B]): M[B]
\end{lstlisting}

Note that this trait must be instantiated on a separate object than the type
\texttt{M} itself, and that object must be provided as an additional argument
to any functions that are monad-agnostic. For example,
\begin{lstlisting}[style=scala]
  def select[A](
      ops: MonadOps,
      mb: ops.M[Boolean],
      left: ops.M[A],
      right: ops.M[A]): ops.M[Unit] =
    ops.bind(mb, (b : Boolean) => if b then left else right)
\end{lstlisting}

Of particular interest is that the function's type signature is able to refer
to the type member \texttt{M} of the argument \texttt{ops}. The ability to use
ad-hoc polymorphism via explicit dictionary-passing in this way is a major
strength of Scala and path-dependent types.

\section{Defunctionalization}

The problem of abstracting over higher-kinded type expressions in a language
with only first-order type variables is well-known. Yallop
\cite{Yallop_White_2014} demonstrated how \emph{defunctionalization} at the
type level could be used to represent higher-kinded programs in OCaml.

Defunctionalization was initially introduced by Reynolds as a technique to
express higher-order \emph{functions} in a first-order language
\cite{Reynolds1972}. To perform this transformation, we replace all occurrences
of first-class functions with regular values, which can be eliminated with a
special \texttt{apply} function. Consider a short program using first-class
functions (figure \ref{fig::defun-pre}).

\begin{figure}
  \begin{lstlisting}[style=scala]
  def map[A, B](f: A => B, l: List[A]) : List[B] =
    l match
      case Nil => Nil
      case Cons(x, xs) => Cons(f(x), map(f, xs))

  val foo = map(((x: Int) => x+1), nums)
  val bar = map(((y: Bool) => !y), bools)
  \end{lstlisting}
  \caption{A simple program using higher-order values}\label{fig::defun-pre}
\end{figure}

To defunctionalize this program, we define a data type with a variant for each
function value used, in this case once for \texttt{foo} and once for
\texttt{bar}. Then, applications are replaced with calls to a specialized
\texttt{apply} function (figure \ref{fig::defun-post}).

\begin{figure}
  \begin{lstlisting}[style=scala]
  sealed trait Fun[A,B]:
    def apply(input: A): B

  object Succ extends Fun[Int, Int]:
    def apply(x: Int): Int = x+1

  object Neg extends Fun[Bool, Bool]:
    def apply(y: Bool): Bool = !y

  def map[A, B](f: Fun[A,B], l: List[A]) : List[B] =
    l match
      case Nil => Nil
      case Cons(x, xs) => Cons(f.apply(x), map(f, xs))

  val foo = map(Succ, nums)
  val bar = map(Neg, bools)
  \end{lstlisting}
  \caption{A defunctionalized program}\label{fig::defun-post}
\end{figure}

The argument to \texttt{map} is now an ordinary, non-function object, meaning
the program as a whole requires no higher-order values\footnote{Classically,
the \texttt{Fun} trait would be a variant type, with the \texttt{apply}
function dispatching via pattern matching instead of inheritance.}!

\subsection{\texttt{higher} in OCaml}

At the type level, things work similarly. Yallop wrote a library, which he
named \texttt{higher}, for higher-kinded programming in OCaml. As we are
re-presenting his work, all code samples in this section will also be in
OCaml.

We introduce an abstract type constructor
\begin{lstlisting}[style=ocaml]
  type ('a, 'f) app
\end{lstlisting}

\noindent
where the type expression \texttt{(s, t) app} represents the application of
the type expression \texttt{t} to the type expression \texttt{s}
\footnote{OCaml type expressions are written "backwards" -- in the type
\texttt{t list}, the argument \texttt{t} comes first. Yallop chose to use the
same convention in the ordering of the arguments to \texttt{app}.}.
Importantly, the expression \texttt{t} here is a proper type -- that is, of
kind~$*$. Type constructors are associated with an uninhabited phantom type,
which Yallop called the \emph{brand}. Beta-reduction does not occur
automatically but is performed explicitly via introduction and eliminator
functions for each brand. For example, the constructor \emph{list} would have a
corresponding module containing the brand \emph{List.t} and projection and
injection functions:
\begin{lstlisting}[style=ocaml]
  module List : sig
    type t
    val inj : 'a list -> ('a, t) app
    val prj : ('a, t) app -> 'a list
  end
\end{lstlisting}

Now, abstraction over type constructors can be expressed by abstracting over
brands, e.g.
\begin{lstlisting}[style=ocaml]
  type 'f monad =
    { pure : `$\forall$`'a . 'a -> ('a, 'f) app
    ; bind : `$\forall$`'a 'b . ('a, 'f) app -> ('a -> ('b, 'f) app) -> ('b, 'f) app
    }
\end{lstlisting}

One concern with defunctionalization is that it is classically a whole-program
transformation -- the single sum type \texttt{app} represents \emph{all}
higher-order type constructors in the program. As an intermediate compiler
representation, this is no obstacle, but \texttt{higher} is intended for use
as an end-user library, and it would not be ergonomic to require that the
programmer know all higher-kinded occurrences up front.

How is the mysterious type \texttt{app} defined? Yallop provided two
implementations and alluded to a third, using an unsafe cast, an extensible
variant type and type families respectively. We detail only the extensible
variant approach, as it most closely resembles the equivalent Scala
construction.

\begin{figure}[ht]
  \begin{lstlisting}[style=ocaml]
  type ('a, 'f) app = ..

  module Newtype(T : sig type 'a t end)() = struct
    type 'a s = 'a T.t
    type t
    type (_, _) app += App : 'a s -> ('a, t) app
    let inj v = App v
    let prj (App v) = v
  end
  \end{lstlisting}
  \caption{Implementing \texttt{higher} with open variants}
    \label{fig::higher-impl}
\end{figure}

Figure \ref{fig::higher-impl} shows this implementation of \texttt{higher}.
The first line declares that \texttt{app} is an open data type whose
constructors will be declared later. Next is the \texttt{Newtype} functor
(a module-level function) which takes, as input, a module declaring a
generic type \texttt{'a t}, which extends the \texttt{app} type with a new
variant \texttt{App} mapping \texttt{'a s} to \texttt{('a, t) app}. The
\texttt{prj} and \texttt{inj} functions then use \texttt{App}. Note that the
match in the \texttt{prj} function is technically inexhaustive, relying on the
fact that \texttt{t} is fresh per invocation of \texttt{Newtype} to guarantee a
match in practice\footnote{It is also important to disallow any downstream
clients from extending \texttt{app} via OCaml's type privacy controls;
otherwise a user may manually define a new constructor with the same type as
\texttt{App}.}.

\subsection{From OCaml to Scala}\label{sec::surface}

In this section, we demonstrate how a similar transformation can be applied in
surface-level Scala. Unlike OCaml, Scala natively supports higher-kinded
polymorphism, so we must ensure that we do not cheat by trivially using the
thing we seek to encode! Specifically, we must ensure that all instances of
polymorphic types \emph{and} abstract type members occur at kind~$*$. Other
forms of first-order polymorphism (namely, polymorphic classes and traits) are
allowed, as these can be expressed in DOT.

The most important part of this translation is the trait \texttt{Apply[F,A]},
corresponding to the OCaml type \texttt{('a, 'f) app}. In OCaml, we used
generalized algebraic datatypes to associate the underlying \texttt{'a T.t}
with the defunctionalized \texttt{('a, t) app}, using pattern matching to
define projection and injection. In Scala, we can use type members for this
purpose:

\begin{lstlisting}[style=scala]
  trait Apply[F,A]:
    type This
    def prj(): This
\end{lstlisting}

This definition has a few key differences from the OCaml implementation.
Instead of encapsulating the type in a module, we attach the projection
function to the object itself. Also of note is the the abstract type member
\texttt{This}, which can refer to the bound parameter \texttt{A} and represents
the fully-realized type. Finally, if the trait is not sealed, we are
automatically extensible, allowing us to easily add new type constructors and
their brands. This is a double-edged sword, as we will see in section
\ref{sec::reflection}.

Instantiations of this trait for \texttt{List} and \texttt{Option} are shown in
figure \ref{fig::instantiation}.

\begin{figure}[ht]
  \begin{minipage}{0.48\textwidth}
    \begin{lstlisting}[style=scala]
  object ListW:
    case object Brand
    type T = Brand.type

    class Inj[A](val x: List[A]) extends Apply[T, A]:
      type This = List[A]
      override def prj(): This = x
    \end{lstlisting}
  \end{minipage}
  \hfill
  \begin{minipage}{0.48\textwidth}
    \begin{lstlisting}[style=scala]
  object OptionW:
    case object Brand
    type T = Brand.type

    class Inj[A](val x: Option[A]) extends Apply[T, A]:
      type This = Option[A]
      override def prj(): This = x
    \end{lstlisting}
  \end{minipage}
  \caption{Two instantiations of \texttt{Apply}.}\label{fig::instantiation}
\end{figure}

Although the inner class \texttt{Inj} takes a parameter, it is allowed because
it is not abstract; the nesting within the outer \texttt{object} is merely for
scoping purposes.

Unfortunately, although these implementations are formulaic, they must be
written by hand. The input to the \texttt{Newtype} functor contains a type
member \texttt{'a t}, which is \emph{exactly} the feature (in Scala) we are
encoding in the first place\footnote{Unlike Scala, the OCaml module language is
separate from its type and expression languages. While OCaml does allow modules
to be used as term-level values, they are far more restricted than Scala's
objects.}! This will be discussed further in section \ref{sec::reflection}.

\subsection{Examples}

We can now easily represent higher-kinded data by abstracting over the brand:
\begin{lstlisting}[style=scala]
  class PersonImpl[F]:
    var name: Apply[F, String]
    var age: Apply[F, Int]

  type UnvalidatedPerson = PersonImpl[OptionW.T]
  type Person = PersonImpl[IdW.T]
\end{lstlisting}

Defining typeclass constraints, like the monad interface, can be done by
further extending \texttt{Apply}:
\begin{lstlisting}[style=scala]
  trait Monad[F, A] extends Apply[F, A]:
    def pure(x: A): Monad[F, A]
    def bind[B](f: A => Monad[F, B]): Monad[F, B]

  def select[M, A](
      mb: Apply[M, Boolean],
      left: Apply[M, A],
      right: Apply[M, A]): Apply[M, Unit] =
    mb.bind((b : Boolean) => if b then left else right)
\end{lstlisting}

Notice that \texttt{select} no longer needs to take the explicit \texttt{ops}
argument, allowing the more idiomatic form \texttt{mb.bind}.

\section{Conclusion}\label{sec::reflection}

We have demonstrated that type-level defunctionalization can be used to
represent higher-kinded polymorphism in Scala while remaining within the
surface area of features expressible in DOT. In this section, we discuss the
potential benefits, alongside further development directions.

\subsection{Limitations in handwritten code}

Compared to \texttt{higher}, type-level defunctionalization in Scala has a
number of downsides making it unsuitable for use by the end-user. This is of no
immediate concern, as Scala supports higher-kinded polymorphism natively (so
end-users have no need to rely on this pattern to express such programs in the
first place), but we outline their limitations below for completeness.

\subsubsection{Brand uniqueness}\label{sec::brands}

The safety of this transformation relies on there being a unique instantiation
of \texttt{Apply[F, A]} for any given brand \texttt{F} and argument \texttt{A}.
However, this is not guaranteed. Consider a violating consumer manually
associating \texttt{Apply[ListW.T, A]} with \texttt{Int} (figure
\ref{fig::bad-brand}).

\begin{figure}[ht]
  \begin{lstlisting}[style=scala]
  object ListW:
    case object Brand
    type T = Brand.type
    // ... (see fig. `\ref{fig::instantiation}`)

  class Bad[A] extends Apply[ListW.T, A]:
    type This = Int
    override def prj(): This = 0
  \end{lstlisting}
  \caption{A careless end-user may break the guarantees of \texttt{Apply}}
  \label{fig::bad-brand}
\end{figure}

In OCaml, the generativity of functors allowed the definition of \texttt{app}
to be hidden from outside consumers, preventing downstream code from generating
new variants other than as allowed by the \texttt{Newtype} functor.
Unfortunately, there is no obvious way to recreate this in Scala with sealed
traits, as we cannot specify the type of the input to a theoretical
\texttt{newtype} function without the use of abstract, higher-kinded type
members.

\subsubsection{Kind checking}\label{sec::kinds}

Similarly, the defunctionalized encoding contains no built-in kind checking of
any sort, limiting the feedback a programmer will receive on ill-kinded uses.
While the trait system can be used to prevent instantiations of ill-kinded
applications like \texttt{Apply[ListW.T,~ListW.T]} by ascribing brands to
generated traits such as, e.g. \texttt{TtoT}, this is cumbersome and
error-prone.

There is also nothing stopping end-users from creating a value of type
\texttt{ListW.T} directly, either by referring to \texttt{ListW.Brand} or via a
typed \texttt{null}. We believe this to be a minor drawback, as it requires
intentional action by the progrmamer that is difficult to perform by accident.

\subsection{As a compiler representation}\label{sec::compiler}

On the other hand, neither issue is a concern if defunctionalization is
performed completely automatically. By using internal, mangled names, a
programmer can never refer to either the trait \texttt{Apply} or witness
objects such as \texttt{ListW} to manually create erroneous instantiations.
Furthermore, the boilerplate of the witness objects lend themselves well to
automation.

Similarly, kind checking can be delegated to the surface language before being
erased in a defunctionalization pass. Ill-kinded snippets such as
\texttt{List[List]} or \texttt{x~:~List} can be ruled out during elaboration,
ensuring that only well-kinded applications remain.

\subsection{Related and future work}

\subsubsection{Implementation details}\label{sec::impl}

This implementation of type-level defunctionalization leans heavily on the
existence of first-order parametric types in the target language. In theory,
this is of little concern, as Amin et al demonstrated that these can be easily
expressed in DOT \cite{amin2016dependent}. In practice, there are still many
design questions left unanswered. The simple, DOT-inspired representation of
generics in dotty was ruled unviable due to expressivity concerns
\cite{OderskyDotty}, and the problematic cases overlap with the construction
presented in this paper. Instead, a defunctionalization pass might be used to
simplify the surface language to a point that new encoding is revealed.

\subsubsection{Relating to DOT}

Stucki et al \cite{Stucki2017HigherOrderSW} presented an extension to
System~F$_\omega$ featuring interval kinds that enforced sub- and supertype
bounds at the kind level, which was sufficient to express many of Scala's
patterns but were unable to extend their results to support full
path-dependence.

One goal of this work is to argue that DOT can express higher-kinded patterns
as-is. However, the translation presented here uses several high-level features
that are not present in DOT, such as traits and rank-1 generics. To rule out a
higher-kinded extension to DOT entirely, a logical next step would be to show a
translation from a higher-kinded extension to DOT to DOT itself. As a first
step, we outline a possible approach in appendix \ref{appendix::dot}.

\subsubsection{Simplification}

As presented in this paper, instantiations of the \texttt{Apply} trait requires
the creation of a separate class \texttt{Inj[A]} to wrap the underlying type.
This is because, unlike Haskell's typeclasses or Rust traits, Scala's traits
cannot be applied to a type post-hoc. If we allow the compiler to insert
further traits to a class at definition time, a far more direct implementation
becomes possible (figure \ref{fig::simple}).
\begin{figure}[ht]
  \begin{lstlisting}[style=scala]
  case object ListW // Compiler-generated, inaccessible in the surface language
  class List[A] extends Apply[ListW, A]:
    // ...
  \end{lstlisting}
  \caption{A simpler way to instantiate \texttt{Apply}}\label{fig::simple}
\end{figure}

By attaching the \texttt{Apply} trait to the underlying class in question, we
obviate the need for explicit projection and injection functions entirely.

This simpler encoding has two major downsides. Firstly, type aliases such as
\texttt{type~Foo[A]~=~String} have no underlying class to which to attach the
trait. Secondly, this encoding admits no obvious way to express a
partially-applied or curried type constructor. If these can be resolved (or
sufficiently worked-around), this may prove to be a superior first-order
representation.

\begin{acks}
  Thanks to Prof. Amin for graciously allowing me to change my final project
  after my initial proposal didn't work.
\end{acks}

\bibliographystyle{ACM-Reference-Format}
\typeout{}
\bibliography{paper}

\begin{appendices}
  \section{Defunctionalization in DOT}\label{appendix::dot}

  In this section, we outline how type-level defunctionalization may be
  performed in DOT proper.

  We leave the specifics of the source language intentionally vague, as its
  design will likely strongly inform the translation. To sufficiently model the
  surface language, it would likely have explicit type-level lambdas and
  associated kinding/subtyping rules.

  Following Amin et al \cite{amin2016dependent}, the translation of the
  \texttt{Apply} trait is straightforward:
  \begin{equation}
    \begin{aligned}
      \texttt{Apply[F, A]} \triangleq \mu (\texttt{self} .
        &(\ &\{ F : * \} \\
        &\land\ &\{ A : * \} \\
        &\land\ &\{ T : \bot \cdot \cdot \{ A : * \} \} \\
        &\land\ &\{ \texttt{prj} : \texttt{self}.T \}))
    \end{aligned}
  \end{equation}

  The translation of types would most likely be kind-directed, rather than
  purely syntactic. This is because a type constructor of arrow kind
  $k_1~\rightarrow~k_2$ must generate a brand and instantiation of
  \texttt{Apply} that can be referred to by the rest of the program, but proper
  types can be translated more straightforwardly. Following Crary
  \cite{CraryFAMC}, we might define the judgment $\Gamma \vdash T:k
  \rightsquigarrow [A.\overline{T}, \overline{T}]$ to separate witness carriers
  (which need to be lifted to the top level) from the final translation (which
  is used in-place). We bind the variable $A$, representing the parameter to
  the lambda, which will eventually be re-bound when instantiating the
  \texttt{Apply} trait at the top level.

  A candidate translation is suggested in figure \ref{fig::sample}.

  \begin{figure}[ht]
    \begin{prooftree}
      \AxiomC{$\Gamma, A: k \vdash T: k' \rightsquigarrow [A'.\overline{T'},
        \overline{T}]$}
      \AxiomC{$\textrm{fresh}(\texttt{brand})$}
      \BinaryInfC{$\Gamma \vdash \lambda (A:k).k' : k \rightarrow k'
        \rightsquigarrow
        [\_. \mu(\texttt{self}.\{ A : * \} \land
          \overline{T} \land
          \overline{T'}[\texttt{self}.A/A']), \texttt{brand}$]}
    \end{prooftree}
    \caption{Sample translation rule for type lambdas}
    \label{fig::sample}
  \end{figure}

  As written, this rule is most likely incorrect; it does not correctly handle
  scoping for nested lambdas, which is crucial for representing curried
  multi-parameter type constructors. Even so, it is hopefully illustrative of
  what the final translation rules may look like.

  A secondary concern is that the correctness of defunctionalization relies on
  the ability to generate infinitely-many fresh brands. While DOT lacks
  direct nominal typing, it can be faked by using fresh label names, which will
  never unify with a pre-existing type in the program.
\end{appendices}
\end{document}
